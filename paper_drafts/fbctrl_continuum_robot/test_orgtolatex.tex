% Created 2016-09-27 Tue 02:14
\documentclass[journal]{IEEEtran}
                          \usepackage[pdftex]{graphicx}
\graphicspath{{../pdf/}{../jpeg/}}
\DeclareGraphicsExtensions{.pdf,.jpeg,.png}
\usepackage[cmex10]{amsmath}
\interdisplaylinepenalty=2500
\usepackage{cite}
\usepackage{epsfig}
\usepackage{epstopdf}
\usepackage[caption=false,font=footnotesize]{subfig}
\usepackage{bm}
\usepackage{color}
\usepackage{hyperref}
\author{Denny Fu}
\date{\today}
\title{}
\hypersetup{
 pdfauthor={Denny Fu},
 pdftitle={},
 pdfkeywords={},
 pdfsubject={},
 pdfcreator={Emacs 24.5.1 (Org mode 8.3.5)}, 
 pdflang={English}}
\begin{document}

\title{Bare Demo of IEEEtran.cls for Journals}

\author{Michael~Shell,~\IEEEmembership{Member,~IEEE,}
John~Doe,~\IEEEmembership{Fellow,~OSA,}
and~Jane~Doe,~\IEEEmembership{Life~Fellow,~IEEE}% <-this % stops a space
\thanks{M. Shell is with the Department
of Electrical and Computer Engineering, Georgia Institute of Technology, Atlanta, GA, 30332 USA e-mail: (see http://www.michaelshell.org/contact.html).}% <-this % stops a space
\thanks{J. Doe and J. Doe are with Anonymous University.}% <-this % stops a space
\thanks{Manuscript received April 19, 2005; revised January 11, 2007.}}


\markboth{Journal of \LaTeX\ Class Files,~Vol.~6, No.~1, January~2007}%
{Shell \MakeLowercase{\textit{et al.}}: Bare Demo of IEEEtran.cls for Journals}

%\IEEEpubid{0000--0000/00\$00.00~\copyright~2007 IEEE}

%\IEEEspecialpapernotice{(Invited Paper)}

\maketitle

\begin{abstract}
%\boldmath
The abstract goes here.
\end{abstract}


\begin{IEEEkeywords}
IEEEtran, journal, \LaTeX, paper, template.
\end{IEEEkeywords}

\IEEEpeerreviewmaketitle

\section{Introduction}
\label{sec:orgheadline3}
\IEEEPARstart{T}{his}
demo file is intended to serve as a ``starter file''
for IEEE journal papers produced under \LaTeX$\backslash$ using
IEEEtran.cls version 1.7 and later.
I wish you the best of success.
\hfill
mds
\hfill
January 11, 2007

\subsection{Subsection Heading Here}
\label{sec:orgheadline2}
Subsection text here.

\subsubsection{Subsection Heading Here}
\label{sec:orgheadline1}
Subsubsection text here.

\section{Conclusion}
\label{sec:orgheadline4}
The conclusion goes here.

\appendices
\section{Proof of the First Zonklar Equation}
\label{sec:orgheadline5}
Appendix one text goes here.

\section{}
\label{sec:orgheadline6}
Appendix two text goes here.


%\bibliographystyle{IEEEtran}
%\bibliography{IEEEabrv,../bib/paper}
\end{document}